\documentclass{article}
\usepackage{sivaSAFRANshort}
\chead{Constructive Approximation Theory: Assignment $1$}
\begin{document}
	\begin{enumerate}
		\item
		Consider the Vandermonde matrix $V$, i.e.,
		$$V = \begin{bmatrix}
		1 & x_0 & x_0^2 & x_0^3 & \cdots & x_0^n\\
		1 & x_1 & x_1^2 & x_1^3 & \cdots & x_1^n\\
		1 & x_2 & x_2^2 & x_2^3 & \cdots & x_2^n\\
		1 & x_3 & x_3^2 & x_3^3 & \cdots & x_3^n\\
		\vdots & \vdots & \vdots & \vdots & \ddots & \vdots\\
		1 & x_n & x_n^2 & x_n^3 & \cdots & x_n^{n}
		\end{bmatrix}$$
		\begin{itemize}
			\item
			Show that $\det(V)$ is a polynomial in the variables $x_0,x_1,\ldots,x_n$ with degree $\dfrac{n(n+1)}2$.
			\item
			Show that if $x_i=x_j$ for $i \neq j$, then $\det(V) = 0$.
			\item
			Hence, conclude that $(x_i-x_j)$ is a factor of $\det(V)$.
			\item
			Hence, conclude that $\det(V) = C \left(\dprod_{1 \leq j <i \leq n} \left(x_i-x_j\right) \right)$, where $C$ is a constant.
			\item
			Compare the coefficient of $x_1x_2^2x_3^3 \cdots x_n^n$ to conclude that $C=1$.
		\end{itemize}
		\item
		Consider uniformly spaced nodes ($x_k = -1+(2k+1)/n$ for $k \in \{0,1,2,\ldots,n-1\}$) and Chebyshev nodes ($y_k= \sin(\pi x_k/2)$ for $k \in \{0,1,2,\ldots,n-1\}$). For both these sets of nodes perform the following:
		\begin{itemize}
			\item
			Plot the condition number of these Vandermonde matrices as a function of $n$. (Use semilogy to plot, i.e., the $Y$ axis is the log(condition number).) Comment on how the condition number scales with $n$.
			\item
			Consider the function $f(x) = \dfrac1{1+25x^2}$. This is called the Runge function. For $n \in \{5,10,20,50\}$, obtain and plot the interpolant by
			\begin{itemize}
				\item
				Solving the linear system
				\item
				Using fundamental Lagrange polynomials, i.e., $\ell_j(x) = \dfrac{\dprod_{k \neq j} (x-x_k)}{\dprod_{k \neq j}(x_j-x_k)}$
			\end{itemize}
			Comment on the interpolant you observe.
			\item
			What is the cost of evaluating the interpolant at a point $x$ as a function of $n$?
			\item
			Based on the above observation, which method would you prefer for polynomial approximation?
		\end{itemize}
		\item
			Show that for any set of interpolation nodes, we have
			$$\dsum_{j=0}^n x_j^m \ell_j(x) = x^m$$
			for all $m \in \{0,1,2,\ldots,n\}$.
		\item
		Recall that a function $f(x)$ on $[-1,1]$ is $\alpha$-H\"{o}lder continuous if for all $x,y \in [-1,1]$, we have $\abs{ f(x)-f(y)} \leq C \abs{x-y}^{\alpha}$ for some $\alpha,C \in \Rb^+$ and is Lipschitz continuous if $\alpha=1$. We will denote $\alpha$-H\"{o}lder continuous functions on $[-1,1]$ as $H^{\alpha}([-1,1])$.
		
		Prove that if $\alpha < \beta$, then $H^{\alpha}([-1,1]) \supset H^{\beta}([-1,1])$.

		\item
		Give examples (with proofs as to why the examples are correct) of function on $[-1,1]$ for the following:
		\begin{itemize}
			\item
			Continuous but not H\"{o}lder continuous for any $\alpha > 0$
			\item
			Lipschitz but not differentiable
			\item
			Differentiable but its derivative is not continuous
		\end{itemize}

	\end{enumerate}
\end{document}